\documentclass[12pt]{article}

\usepackage{listings}
\usepackage{graphicx}
\usepackage{float}

\usepackage{amssymb} 
\usepackage{amsmath}
\usepackage{amsthm}
\usepackage{sansmath}
\usepackage{epsfig}

\usepackage[T1]{fontenc}
\usepackage{lmodern}
\renewcommand*\familydefault{\sfdefault}

\usepackage[margin=2cm]{geometry}

\usepackage[parfill]{parskip}
\linespread{1.2} 

\usepackage{xcolor}

\lstdefinestyle{code}{  
    commentstyle=\color{gray},
    keywordstyle=\color{blue},
    numberstyle=\tiny\color{gray},
    stringstyle=\color{teal},
    ndkeywordstyle=\color{purple},
    basicstyle=\ttfamily\footnotesize,
    breakatwhitespace=false,         
    breaklines=true,                 
    captionpos=t,                    
    keepspaces=true,                 
    numbers=left,                    
    numbersep=8pt,                  
    showspaces=false,                
    showstringspaces=false,
    showtabs=false,                  
    tabsize=4,
    frame=single,
    breaklines=true
}

\lstset{style=code}

\newcounter{question}
\setcounter{question}{0}
\newcounter{subquest}

\newcommand{\question}[1]{
    \stepcounter{question} 
    \vspace{1em}
    \textbf{\Large\thequestion \ - #1}
    \vspace{.5em} 
    \setcounter{subquest}{0}\ \\}

\newcommand{\subquestion}{
    \stepcounter{subquest} 
    \vspace{.5em}
    \textbf{\large Question \thequestion.\thesubquest}
    \vspace{.25em}\ \\}

\newcommand{\solution}
    {\par\vspace{0.5em}\noindent\emph{Solution.}\ }
    {\par\vspace{1em}}

\usepackage{fancyhdr}
\setlength{\headheight}{15pt}
\pagestyle{fancy} 
\lhead{Math 5334: \emph{Numerical Analysis}}
\rhead{Serena Su}
\cfoot{}
\rfoot{\thepage}

\begin{document}

\begin{center}
\textbf{\huge Assignment X}

% {\Large{Coding Documentation}}

{\large\emph{Due Date: XXXX}}
\vspace{1em}
\end{center}
\hrule

% For Code Documentation
% \section*{Introduction}

% lorem ipsum.

% \section*{Cases Handled}
% The following cases are considered:
% \begin{itemize}
%     \item Case 1
%     \item Case 2
% \end{itemize}

% \section*{Algorithm}
% The algorithm follows these general steps:
% \begin{enumerate}
%     \item Case 1
%     \begin{enumerate}
%         \item Subcase A
%     \end{enumerate}
% \end{enumerate}

% \section*{Function Documentation}
% \subsection*{Main Function\\} 
% \hrule
% {\large\texttt{function(param)}}

% \textbf{Description:}

% lorem ipsum.

% \begin{description}
%     \item[Parameters:] 
%     \item \texttt{param} : \texttt{(type)} lorem ipsum.

%     \item[Returns:] 
%     \item \texttt{type :} lorem ipsum.

%     \item[Raises:] 
%     \item \texttt{Error:} lorem ipsum. \\
% \end{description}

% \hrule

% \subsection*{Helper Functions\\}
% \hrule
% {\large\texttt{function(param)}}

% \textbf{Description:}

% lorem ipsum.

% \begin{description}
%     \item[Parameters:] 
%     \item \texttt{param} : \texttt{(type)} lorem ipsum.

%     \item[Returns:] 
%     \item \texttt{type :} lorem ipsum.

%     \item[Raises:] 
%     \item \texttt{Error:} lorem ipsum. \\
% \end{description}
% \hrule

\end{document}