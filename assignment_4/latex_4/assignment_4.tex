\documentclass[12pt]{article}

\usepackage{listings}
\usepackage{graphicx}
\usepackage{float}
\usepackage{hyperref}
\usepackage{ifthen}
\usepackage{pdfpages}

\usepackage{amssymb} 
\usepackage{amsmath}
\usepackage{amsthm}
\usepackage{sansmath}
\usepackage{epsfig}

\usepackage[T1]{fontenc}
\usepackage{lmodern}
\renewcommand*\familydefault{\sfdefault}

\usepackage[margin=2cm]{geometry}

\usepackage[parfill]{parskip}
\linespread{1.2} 

\usepackage{xcolor}

\lstdefinestyle{code}{  
    commentstyle=\color{gray},
    keywordstyle=\color{blue},
    numberstyle=\tiny\color{gray},
    stringstyle=\color{teal},
    ndkeywordstyle=\color{purple},
    basicstyle=\ttfamily\footnotesize,
    breakatwhitespace=false,         
    breaklines=true,                 
    captionpos=t,                    
    keepspaces=true,                 
    numbers=left,                    
    numbersep=8pt,                  
    showspaces=false,                
    showstringspaces=false,
    showtabs=false,                  
    tabsize=4,
    frame=single,
    breaklines=true
}

\lstset{style=code}

\newcounter{question}
\setcounter{question}{0}
\newcounter{subquest}

\newcommand{\question}{
    \stepcounter{question} 
    \newpage
    \vspace{1.5em}
    \textbf{\\ \Large Question \thequestion \ }
    \vspace{.5em} 
    \setcounter{subquest}{0}\ \\}

\newcommand{\subquestion}[1][true]{
    \stepcounter{subquest} 
    \ifthenelse{\equal{#1}{true} \and \value{subquest}>1}{\newpage}{}
    \vspace{1em}
    \textbf{\large Question \thequestion.\thesubquest}
    \vspace{.5em}\ \\}

\newcommand{\solution}
    {\par\vspace{0.5em}\noindent\emph{Solution.}\ }
    {\par\vspace{1em}}

\newcommand{\startappendix}{%
    The complete code used for this assignment is provided in the appendix for reference. Files can be accessed directly at this \href{https://github.com/sjsusu/numerical_analysis}{GitHub repository}.
}

\usepackage{fancyhdr}
\setlength{\headheight}{15pt}
\pagestyle{fancy} 
\lhead{Math 5334: \emph{Numerical Analysis}}
\rhead{Serena Su}
\cfoot{}
\rfoot{\thepage}

\begin{document}

\begin{center}
\textbf{\huge Homework 4}

{\large\emph{Due Date: November 21, 2025}}
\vspace{1em}
\end{center}
\hrule

\vspace{1em}
\textbf{\Large 2D Heat Equation with Finite Element Method}

Consider the time-dependent heat equation as follows:
\[\frac{\partial u}{\partial t} - \nu \Delta u = f(x,y,t) \quad \text{in} \quad \Omega = (-2,2) \times (-2,2), \quad t \in (0,1]\]
with diffusion coefficient $\nu = 0.05$ and corresponding homogeneous Dirichlet boundary conditions:
\[u(x,y,\cdot) = 0 \quad \text{for} \quad (x,y) \in \partial \Omega\]
We assume the exact solution is given by:
\[u_{exact}(x,y,t) = e^{-8\pi^2 \nu t} \sin(2\pi x) \sin(2\pi y)\]
We will be using rectangular elements. You will use the bilinear $Q_1$ element as your basis function to solve the heat equation. The corresponding four shape functions defined in the reference element $(\xi,\eta) \in (-1,1) \times (-1,1)$ are:
\[\Phi_1(\xi,\eta) = \frac{1}{4}(1-\xi)(1-\eta), \quad \Phi_2(\xi,\eta) = \frac{1}{4}(1+\xi)(1-\eta),\]
\[\Phi_3(\xi,\eta) = \frac{1}{4}(1+\xi)(1+\eta), \quad \Phi_4(\xi,\eta) = \frac{1}{4}(1-\xi)(1+\eta)\]

\newpage
{\Large \textbf{Preliminary Setup}}\\
This is a general outline of the finite element method for solving the 2D heat equation. The specifics for our problem will be addressed in the subsequent questions.

\vspace{1em}
{\large \textbf{Weak Formulation}} \\
Let $v$ be a test function belonging to the function space:
\[V = \{v \in H^1_0(\Omega) \ | \ v, v' \in L^2(\Omega)\}\]
Note that $v=0$ on $\partial \Omega$. Multiplying the PDE by $v$ and integrating over the domain $\Omega$, we have:
\[\int_{\Omega} u_t v - \nu v \Delta u \ dx = \int_{\Omega} f v \ dx\]
Integrating by parts on the second term of the left-hand side:
\[\int_{\Omega} u_t v \ dx + \nu [\int_{\Omega} \nabla u \cdot \nabla v \ dx - \int_{\partial \Omega} v (\nabla u \cdot n) \ ds] = \int_{\Omega} f v \ dx\]
Since $v=0$ on $\partial \Omega$, the boundary integral vanishes. Therefore, the weak formulation is given by:
\[\int_{\Omega} u_t v \ dx + \nu \int_{\Omega} \nabla u \cdot \nabla v \ dx = \int_{\Omega} f v \ dx\]
Where $u,v \in V$. Denote the left hand side as $a(u,v)$ and the right hand side as $L(v)$.

\vspace{1em}
{\large \textbf{Discretization and Global System}} \\
We discretize the spatial domain $\Omega$ into rectangular elements. Let $V_h \subset V$ be the finite-dimensional subspace spanned by the basis functions $\{\phi_i\}_{i=1}^{N}$, where $N$ is the total number of nodes in the mesh (will be reduced later on based on BC). Each bilinear $\phi_n$ corresponds to some node $(x_i, y_j)$ satisfying $\phi_{i,j} = \delta_{i,j}$. We assume $u_h\in V_h$ satisfies the weak formulation $a(u_h,v_h) = L(v_h)$ for all $v_h \in V_h$.

Approximate the solution of $u$ as:
\[u_h = \sum_{j=1}^{N} U_j(t) \phi_j(x,y)\]
where $U_j$ are time-dependent coefficients to be determined. The test function $v$ is also chosen from the same space and discretized similarly:
\[v_h = \sum_{i=1}^{N} V_i(t) \phi_i(x,y)\]
Substituting these approximations into the weak formulation, we obtain the system:
\[\sum_{i,j=1}^{N} V_j \left( \int_{\Omega} \phi_i \phi_j \ dx \right) \frac{dU_i}{dt} + \nu \sum_{i,j=1}^{N} V_j \left( \int_{\Omega} \nabla \phi_i \cdot \nabla \phi_j \ dx \right) U_i = \sum_{j=1}^{N}V_j\int_{\Omega} f \phi_j \ dx\]
We may factor out the $V_j$ to give us: 
\[\sum_{i,j=1}^{N} \left( \int_{\Omega} \phi_i \phi_j \ dx \right) \frac{dU_i}{dt} + \nu \sum_{i,j=1}^{N}  \left( \int_{\Omega} \nabla \phi_i \cdot \nabla \phi_j \ dx \right) U_i = \sum_{j=1}^{N}\int_{\Omega} f \phi_j \ dx\] 
More compactly, let:
\[ U = [U_1, U_2, \ldots, U_N]^T, \quad M_{ij} = \int_{\Omega} \phi_i \phi_j \ dx, \quad K_{ij} = \int_{\Omega} \nabla \phi_i \cdot \nabla \phi_j \ dx, \quad F_j = \int_{\Omega} f \phi_j \ dx\]
such that $M = (M_{ij})$, $K = (K_{ij})$, and $F = [F_1, F_2, \ldots, F_N]^T$.
We can rewrite the system as:
\[M \frac{dU}{dt} + \nu K U = F\]
where $M$ is the mass matrix, $K$ is the stiffness matrix, and $F$ is the force vector.

\vspace{1em}
{\large \textbf{Elemental-Level Systems and Assembly}} \\
We suppose element-wise, each $u^{(e)}$ satisfies the weak formulation over its own domain $\Omega^{(e)}$:
\[\int_{\Omega^{(e)}} u_t^{(e)} v^{(e)} \ dx + \nu \int_{\Omega^{(e)}} \nabla u^{(e)} \cdot \nabla v^{(e)} \ dx = \int_{\Omega^{(e)}} f v^{(e)} \ dx\]
We suppose the local approximations are given by:
\[u_h^{(e)} = \sum_{j=1}^{4} U_j^{(e)} \phi_j^{(e)}(x,y), \quad v_h^{(e)} = \sum_{i=1}^{4} V_i^{(e)} \phi_i^{(e)}(x,y)\]
since we have rectangular elements with four nodes each.
Using the same process as before, we can derive the elemental system:
\[M^{(e)} \frac{dU^{(e)}}{dt} + \nu K^{(e)} U^{(e)} = F^{(e)}\]
We can express the global matrices and vector as sums over all elements:
\[M = \sum_{e=1}^{E} M^{(e)}, \quad K = \sum_{e=1}^{E} K^{(e)}, \quad F = \sum_{e=1}^{E} F^{(e)}\]
where $E$ is the total number of elements, and the elemental matrices and vector are defined as:
\[M_{ij}^{(e)} = \int_{\Omega^{(e)}} \phi_i^{(e)} \phi_j^{(e)} \ dx, \quad K_{ij}^{(e)} = \int_{\Omega^{(e)}} \nabla \phi_i^{(e)} \cdot \nabla \phi_j^{(e)} \ dx, \quad F_j^{(e)} = \int_{\Omega^{(e)}} f \phi_j^{(e)} \ dx\]
Here, $\Omega^{(e)}$ is the domain of element $e$, and $\phi_i^{(e)}$ are the local shape functions associated with element $e$.

We can use quadrature to numerically compute the integrals for $M^{(e)}$, $K^{(e)}$, and $F^{(e)}$ on each element, then assemble them into the global system.

\newpage
\question
By substituting $u_{exact}$ into the PDE, determine the forcing term $f(x,y,t)$ such that:
\[\frac{\partial u_{exact}}{\partial t}- \nu \Delta u_{exact} = f(x,y,t)\]

\solution
For the purposes of this question, denote $u = u_{exact}$. Where $u$ is the exact solution given by:
\[u(x,y,t) = e^{-8\pi^2 \nu t} \sin(2\pi x) \sin(2\pi y)\]
Let us first compute the time derivative:
\[\frac{\partial u}{\partial t} = -8\pi^2 \nu e^{-8\pi^2 \nu t} \sin(2\pi x) \sin(2\pi y)\]
Next, we want to find the Laplacian. Computing the first and second derivative with respect to $x$:
\[\frac{\partial u}{\partial x} = 2\pi e^{-8\pi^2 \nu t} \cos(2\pi x) \sin(2\pi y)\]
\[\frac{\partial^2 u}{\partial x^2} = -4\pi^2 e^{-8\pi^2 \nu t} \sin(2\pi x) \sin(2\pi y)\]
The 2nd derivative with respect to $y$ is the same:
\[\frac{\partial^2 u}{\partial y^2} = -4\pi^2 e^{-8\pi^2 \nu t} \sin(2\pi x) \sin(2\pi y)\]
Therefore, the Laplacian is:
\[\Delta u = \frac{\partial^2 u}{\partial x^2} + \frac{\partial^2 u}{\partial y^2} = -8\pi^2 e^{-8\pi^2 \nu t} \sin(2\pi x) \sin(2\pi y)\]
Substituting these results into the heat equation, we have:
\[\frac{\partial u}{\partial t} - \nu \Delta u = -8\pi^2 \nu e^{-8\pi^2 \nu t} \sin(2\pi x) \sin(2\pi y) - \nu (-8\pi^2 e^{-8\pi^2 \nu t} \sin(2\pi x) \sin(2\pi y)) = 0\]
So our forcing term is:
\[f(x,y,t) = 0\]
and we are working with the homogeneous heat equation.

\question
Discretize the spacial domain $\Omega$ into $16$ equal square elements arranged in a $4 \times 4$ grid, with the node coordinates:
\[(x,y) \in \{-2,-1,0,1,2\} \times \{-2,-1,0,1,2\}\]
Draw this mesh, define your own global numbering, label all global node numbers, and generate corresponding elemental connectivities.

\solution
All indexing used will start from $0$. The mesh is as follows:
\begin{center}
\includegraphics[width=.75\textwidth]{mesh.jpg}
\end{center}

Globally, we have $25$ nodes numbered from $0$ to $24$. Numbering starts from the top-left corner and goes row-wise. The elements are numbered from $0$ to $15$, also row-wise in the same fashion. For each element, its indicies $(0, 1,2,3)$ start from the bottom-left and go counter-clockwise to match the definition of the bilinear $Q1$ element. 
The folowing code was used to generate the connectivity matrix:
\begin{lstlisting}[language=Python, caption={Question 2 Code}]
import numpy as np
import sympy as sp

def global_indexing(width, height=None, include_boundary=False):
    if height is None:
        height = width
    if include_boundary:
        return np.arange(width * height).reshape((width, height))
    return np.arange((width-2)*(height-2)).reshape((width-2, height-2))

def generate_connectivity_matrix(global_indices):
    total_elements = (global_indices.shape[0] - 1) * (global_indices.shape[1] - 1)
    connectivity_matrix = np.zeros((total_elements, 4), dtype=int)
    element = 0
    for i in range(global_indices.shape[0] - 1):
        for j in range(global_indices.shape[1] - 1):
            connectivity_matrix[element, 0] = global_indices[i+1, j]
            connectivity_matrix[element, 1] = global_indices[i+1, j+1]
            connectivity_matrix[element, 2] = global_indices[i, j+1]
            connectivity_matrix[element, 3] = global_indices[i, j]
            element += 1
    return connectivity_matrix

if __name__ == "__main__":
    width = 5  # Number of nodes along one dimension
    global_with_boundary = global_indexing(width, include_boundary=True)
    connectivity_matrix_with_boundary = generate_connectivity_matrix(global_with_boundary)

    with open("./outputs_4/matrices.txt", "w") as f:
        latex_matrix = sp.latex(sp.Matrix(connectivity_matrix_with_boundary))
        f.write("Connectivity Matrix with Boundary:\n")
        f.write(latex_matrix + "\n\n")
\end{lstlisting}
The elemental connectivities are as follows:
\begin{center}
    \begin{tabular}{c| c c c c}
        Element & Node 0 & Node 1 & Node 2 & Node 3 \\
        \hline
        0 & 5 & 6 & 1 & 0 \\
        1 & 6 & 7 & 2 & 1 \\
        2 & 7 & 8 & 3 & 2 \\
        3 & 8 & 9 & 4 & 3 \\
        4 & 10 & 11 & 6 & 5 \\
        5 & 11 & 12 & 7 & 6 \\
        6 & 12 & 13 & 8 & 7 \\
        7 & 13 & 14 & 9 & 8 \\
        8 & 15 & 16 & 11 & 10 \\
        9 & 16 & 17 & 12 & 11 \\
        10 & 17 & 18 & 13 & 12 \\
        11 & 18 & 19 & 14 & 13 \\
        12 & 20 & 21 & 16 & 15 \\
        13 & 21 & 22 & 17 & 16 \\
        14 & 22 & 23 & 18 & 17 \\
        15 & 23 & 24 & 19 & 18 \\
        
    \end{tabular}
\end{center}

\question
For one physical element $u^{(e)}$, write the mapping from the reference element $(\xi,\eta) \in (-1,1) \times (-1,1)$ to the physical coordinates $(x,y)$ in terms of the nodal coordinates $(x_n, y_n)$ and the shape functions $\Phi_n(\xi,\eta)$. Also, derive the Jacobian matrix $J(\xi,\eta)$ of this mapping. 

\solution
Reindexing the four shape functions to fit our indexing scheme for the element nodes, we have:
\[\Phi_0(\xi,\eta) = \frac{1}{4}(1-\xi)(1-\eta), \quad \Phi_1(\xi,\eta) = \frac{1}{4}(1+\xi)(1-\eta),\]
\[\Phi_2(\xi,\eta) = \frac{1}{4}(1+\xi)(1+\eta), \quad \Phi_3(\xi,\eta) = \frac{1}{4}(1-\xi)(1+\eta)\]
On a physical element $e$, it is a rectangle of the region $[x_0, x_1] \times [y_0, y_1]$ where $(x_0, y_0)$ is the bottom-left corner and $(x_1, y_1)$ is the top-right corner. The mapping from $(\xi,\eta) \mapsto (x,y)$ should be given by the standard change of variables:
\[\begin{bmatrix}
x(\xi,\eta) \\y(\xi,\eta)
\end{bmatrix}
=
\begin{bmatrix}
\frac{1}{2}(x_1 - x_0) \xi + \frac{1}{2}(x_1 + x_0) \\
\frac{1}{2}(y_1 - y_0) \eta + \frac{1}{2}(y_1 + y_0)
\end{bmatrix} \label{(1)} \tag{1}\]
We will verify this using the shape functions. We assume that:
\[x(\xi,\eta) = \sum_{n=0}^{3} x_n^{(e)} \Phi_n^{(e)}(\xi,\eta), \quad y(\xi,\eta) = \sum_{n=0}^{3} y_n^{(e)} \Phi_n^{(e)}(\xi,\eta)\]
where $(x_n^{(e)}, y_n^{(e)})$ are the nodal coordinates of element $e$. Note that by our indexing scheme:
\[(x_0^{(e)}, y_0^{(e)}) = (x_0, y_0), \quad (x_1^{(e)}, y_1^{(e)}) = (x_1, y_0),\]
\[(x_2^{(e)}, y_2^{(e)}) = (x_1, y_1), \quad (x_3^{(e)}, y_3^{(e)}) = (x_0, y_1)\]
Expanding $x(\xi,\eta)$:
\begin{align*}
x(\xi,\eta) &= x_0^{(e)} \Phi_0^{(e)} + x_1^{(e)} \Phi_1^{(e)} + x_2^{(e)} \Phi_2^{(e)} + x_3^{(e)} \Phi_3^{(e)}\\
&= x_0 \frac{1}{4}(1-\xi)(1-\eta) + x_1 \frac{1}{4}(1+\xi)(1-\eta) + x_1 \frac{1}{4}(1+\xi)(1+\eta) + x_0 \frac{1}{4}(1-\xi)(1+\eta)\\
&= \frac{1}{4} \left[ x_0(1-\xi)(1-\eta +1+\eta) + x_1(1+\xi)(1-\eta + 1+\eta) \right]\\
&= \frac{1}{4} \left[ 2x_0(1-\xi) + 2x_1(1+\xi) \right]\\
&= \frac{x_1 - x_0}{2} \xi + \frac{x_1 + x_0}{2}
\end{align*}
Similarly for $y(\xi,\eta)$, we get:
\[y(\xi,\eta) = \frac{y_1 - y_0}{2} \eta + \frac{y_1 + y_0}{2}\]
Therefore, the mapping from reference to physical coordinates is given by equation \ref{(1)}

Now we can derive the Jacobian matrix, $J(\xi,\eta)$, of this mapping, which is defined as:
\[J(\xi,\eta) = \begin{bmatrix}
\frac{\partial x}{\partial \xi} & \frac{\partial x}{\partial \eta} \\
\frac{\partial y}{\partial \xi} & \frac{\partial y}{\partial \eta}
\end{bmatrix}\]
Let us compute each partial derivative:
\[\frac{\partial x}{\partial \xi} = \frac{x_1 - x_0}{2}, \quad \frac{\partial x}{\partial \eta} = 0\]
\[\frac{\partial y}{\partial \xi} = 0, \quad \frac{\partial y}{\partial \eta} = \frac{y_1 - y_0}{2}\]
Therefore, our Jacobian is:
\[J(\xi,\eta) = \begin{bmatrix}
\frac{x_1 - x_0}{2} & 0 \\ 
0 & \frac{y_1 - y_0}{2}
\end{bmatrix}\]
Note that if our elements were all equally sized, the Jacobian would be identical for all elements.

\question
Using the basis functions from the reference element, compute the following derivatives:
\[\frac{\partial \Phi_i}{\partial \xi}, \frac{\partial \Phi_i}{\partial \eta} \quad \text{for} \quad i=0,1,2,3\]
Then express the physical gradients $\nabla \Phi_i = 
\begin{bmatrix}\frac{\partial \Phi_i}{\partial x}, \frac{\partial \Phi_i}{\partial y}\end{bmatrix}^T$ using the Jacobian.

\solution
Note that:
\[\Phi_i(x,y) = \Phi_i(\xi(x,y), \eta(x,y))\]
We know that:
\[\nabla \Phi_i(\xi,\eta) = J(\xi,\eta) \ \nabla \Phi_i(x,y)\]
where $J$ is the Jacobian matrix derived in the previous question. Therefore, we can express the physical gradients as:
\[\nabla \Phi_i(x,y) = J^{-1}(\xi,\eta) \ \nabla \Phi_i(\xi,\eta)\]
Given the Jacobian from before:
\[J(\xi,\eta) = \begin{bmatrix}
\frac{x_1 - x_0}{2} & 0 \\ 
0 & \frac{y_1 - y_0}{2}
\end{bmatrix}\]
Its inverse is given by:
\[J^{-1}(x,y) = \begin{bmatrix}
\frac{2}{x_1 - x_0} & 0 \\ 
0 & \frac{2}{y_1 - y_0}
\end{bmatrix}\]

Then for each $\Phi_i(x,y)$, we have:
\begin{align*}
    \nabla \Phi_i(x,y) &= J^{-1}(\xi,\eta) \ \nabla \Phi_i(\xi,\eta) \\
    &= \begin{bmatrix}
\frac{2}{x_1 - x_0} & 0 \\ 
0 & \frac{2}{y_1 - y_0}
\end{bmatrix} \begin{bmatrix}
\frac{\partial \Phi_i}{\partial \xi} \\ \frac{\partial \Phi_i}{\partial \eta}
\end{bmatrix} \\
&= \begin{bmatrix}
\frac{2}{x_1 - x_0} & 0 \\ 
0 & \frac{2}{y_1 - y_0}
\end{bmatrix} \begin{bmatrix}
\frac{\partial \Phi_i}{\partial \xi} \\ \frac{\partial \Phi_i}{\partial \eta}
\end{bmatrix} \\
&= \begin{bmatrix}
\frac{2}{x_1 - x_0} \frac{\partial \Phi_i}{\partial \xi} \\ \frac{2}{y_1 - y_0} \frac{\partial \Phi_i}{\partial \eta}
\end{bmatrix}
\end{align*}

Since we have square elements of length $1$, we have $x_1 - x_0 = 1$ and $y_1 - y_0 = 1$. Therefore, the physical gradients simplify to:
\[\nabla \Phi_i(x,y) = 2 \nabla \Phi_i(\xi,\eta)\]

Let us first calculate $\nabla \Phi_i(\xi,\eta)$. For reference, the shape functions are:
\[\Phi_0(\xi,\eta) = \frac{1}{4}(1-\xi)(1-\eta), \quad \Phi_1(\xi,\eta) = \frac{1}{4}(1+\xi)(1-\eta),\]
\[\Phi_2(\xi,\eta) = \frac{1}{4}(1+\xi)(1+\eta), \quad \Phi_3(\xi,\eta) = \frac{1}{4}(1-\xi)(1+\eta)\]

Starting with derivatives with respect to $\xi$:
\[\frac{\partial \Phi_0}{\partial \xi} = -\frac{1}{4}(1-\eta), \quad \frac{\partial \Phi_1}{\partial \xi} = \frac{1}{4}(1-\eta),\]
\[\frac{\partial \Phi_2}{\partial \xi} = \frac{1}{4}(1+\eta), \quad \frac{\partial \Phi_3}{\partial \xi} = -\frac{1}{4}(1+\eta)\]

Then for derivatives with respect to $\eta$:
\[\frac{\partial \Phi_0}{\partial \eta} = -\frac{1}{4}(1-\xi), \quad \frac{\partial \Phi_1}{\partial \eta} = -\frac{1}{4}(1+\xi),\]
\[\frac{\partial \Phi_2}{\partial \eta} = \frac{1}{4}(1+\xi), \quad \frac{\partial \Phi_3}{\partial \eta} = \frac{1}{4}(1-\xi)\]

Therefore, the gradients in reference coordinates are:
\[\nabla \Phi_0(\xi,\eta) = \frac{1}{4}\begin{bmatrix}
-(1-\eta) \\ -(1-\xi)
\end{bmatrix}, \quad \nabla \Phi_1(\xi,\eta) = \frac{1}{4}\begin{bmatrix}
(1-\eta) \\ -(1+\xi)
\end{bmatrix}\] 
\[\nabla \Phi_2(\xi,\eta) = \frac{1}{4}\begin{bmatrix}
(1+\eta) \\ (1+\xi)
\end{bmatrix}, \quad \nabla \Phi_3(\xi,\eta) = \frac{1}{4}\begin{bmatrix}
-(1+\eta) \\ (1-\xi)
\end{bmatrix}\]
So we have the physical gradients:
\[\nabla  \Phi_0(x,y) = \frac{1}{2}\begin{bmatrix}
-(1-\eta) \\ -(1-\xi)
\end{bmatrix}, \quad \nabla  \Phi_1(x,y) = \frac{1}{2}\begin{bmatrix}
(1-\eta) \\ -(1+\xi)
\end{bmatrix}\] 
\[\nabla  \Phi_2(x,y) = \frac{1}{2}\begin{bmatrix}
(1+\eta) \\ (1+\xi)
\end{bmatrix}, \quad \nabla  \Phi_3(x,y) = \frac{1}{2}\begin{bmatrix}
-(1+\eta) \\ (1-\xi)
\end{bmatrix}\]


\question
Use the following formulas for the elemental mass matrix $M^{(e)}$, and stiffness matrix $K^{(e)}$:
\[M_{ij}^{(e)} = \int_{\Omega^{(e)}} \Phi_i^{(e)} \Phi_j^{(e)} \ dx, \quad K_{ij}^{(e)} = \int_{\Omega^{(e)}} \nabla \Phi_i^{(e)} \cdot \nabla \Phi_j^{(e)} \ dx\]
to evaluate these integrals explicitly for an arbitrary square element in this mesh. Your $M^{(e)}$ and $K^{(e)}$ should be $4 \times 4$ matrices.

\solution
For ease of notation let $\Phi_i^{(e)} = \Phi_i$.

Let us denote $\Phi = [\Phi_0, \Phi_1, \Phi_2, \Phi_3]^T$ as the vector of shape functions for element $e$. Note that $M^{(e)}$ can also be expressed a $M^{(e)} = \int_{\Omega_e} \Phi \cdot \Phi^T \ dx$. Using the change of variables from physical to reference coordinates, we have:
\[M^{(e)} = \int_{-1}^{1} \int_{-1}^{1} \Phi \cdot \Phi^T |\det{J(\xi,\eta)}| \ d\xi d\eta\]
where $|\det{J(\xi,\eta)}|$ is the absolute value of the determinant of our Jacobian. From question 5, we have:
\[|\det{J(\xi,\eta)}| = \left| \frac{(x_1-x_0)}{2} \cdot \frac{(y_2-y_1)}{2}\right| = \frac{|x_1-x_0| \cdot |y_2-y_1|}{4}\]
When corrected for our local indexing scheme.

Since this constant, we can factor it out of the integral:
\[M^{(e)} = \frac{|x_1-x_0| \cdot |y_2-y_1|}{4} \int_{-1}^{1} \int_{-1}^{1} \Phi \cdot \Phi^T \ d\xi d\eta\]
Note that:
\[\Phi \cdot \Phi^T = \begin{bmatrix} 
    \Phi_0 \\ \Phi_1 \\ \Phi_2 \\ \Phi_3
\end{bmatrix}
\begin{bmatrix}
    \Phi_0 & \Phi_1 & \Phi_2 & \Phi_3
\end{bmatrix}\]
\[= \begin{bmatrix}
\Phi_0 \Phi_0 & \Phi_0 \Phi_1 & \Phi_0 \Phi_2 & \Phi_0 \Phi_3 \\
\Phi_1 \Phi_0 & \Phi_1 \Phi_1 & \Phi_1 \Phi_2 & \Phi_1 \Phi_3 \\
\Phi_2 \Phi_0 & \Phi_2 \Phi_1 & \Phi_2 \Phi_2 & \Phi_2 \Phi_3 \\
\Phi_3 \Phi_0 & \Phi_3 \Phi_1 & \Phi_3 \Phi_2 & \Phi_3 \Phi_3
\end{bmatrix}\]
We will compute some auxiliary integrals with dummy variables first:
\[\int_{-1}^{1} (1\pm z)^2 \ dz = \pm \frac{(1\pm z)^3}{3} \Big|_{-1}^{1} = \frac{8}{3}\]
\[\int_{-1}^1 (1+z)(1-z) \ dz = \int_{-1}^1 (1 - z^2) \ dz = z - \frac{z^3}{3} \Big|_{-1}^{1} = \frac{4}{3}\]
By the nature of the calculations, the integral matrix will be symmetric. We will show one sample calculation for each main-diagonal and off-diagonal entries. 
For reference, the shape functions are:
\[\Phi_0(\xi,\eta) = \frac{1}{4}(1-\xi)(1-\eta), \quad \Phi_1(\xi,\eta) = \frac{1}{4}(1+\xi)(1-\eta),\]
\[\Phi_2(\xi,\eta) = \frac{1}{4}(1+\xi)(1+\eta), \quad \Phi_3(\xi,\eta) = \frac{1}{4}(1-\xi)(1+\eta)\]
First note that the product of any pair of shape functions will always have a factor of $\frac{1}{16}$. Also note that each product pair will contain two factors in some combination of the forms shown in the auxiliary integrals above.

Case: Main-diagonal entry ($i=j=0$):
\begin{align*}
\int_{-1}^{1} \int_{-1}^{1} \Phi_0 \Phi_0 \ d\xi d\eta &= \int_{-1}^{1} \int_{-1}^{1} \frac{1}{16}(1-\xi)^2(1-\eta)^2 \ d\xi d\eta \\
&= \frac{1}{16} \int_{-1}^{1} (1-\eta)^2 \int_{-1}^1 (1-\xi)^2 \ d\xi d\eta \\
&= \frac{1}{16} \cdot \frac{8}{3} \int_{-1}^{1} (1-\eta)^2 \ d\eta \\
&= \frac{1}{16} \cdot \frac{8}{3} \cdot \frac{8}{3}\\
&= \frac{4}{9}
\end{align*}
Case: 1st Off-diagonal entry ($i=0, j=1$):
\begin{align*}
\int_{-1}^{1} \int_{-1}^{1} \Phi_0 \Phi_1 \ d\xi d\eta &= \int_{-1}^{1} \int_{-1}^{1} \frac{1}{16}(1-\xi)(1-\eta)(1+\xi)(1-\eta) \ d\xi d\eta \\
&= \frac{1}{16} \int_{-1}^{1} (1-\eta)^2 \int_{-1}^1 (1-\xi^2) \ d\xi d\eta \\
&= \frac{1}{16} \cdot \frac{4}{3} \int_{-1}^{1} (1-\eta)^2 \ d\eta \\
&= \frac{1}{16} \cdot \frac{4}{3} \cdot \frac{8}{3}\\
&= \frac{2}{9}
\end{align*}
Case: 2nd Off-diagonal entry ($i=0, j=2$):
\begin{align*}
\int_{-1}^{1} \int_{-1}^{1} \Phi_0 \Phi_2 \ d\xi d\eta &= \int_{-1}^{1} \int_{-1}^{1} \frac{1}{16}(1-\xi)(1-\eta)(1+\xi)(1+\eta) \ d\xi d\eta \\
&= \frac{1}{16} \int_{-1}^{1} (1-\eta^2) \int_{-1}^1 (1-\xi^2) \ d\xi d\eta \\
&= \frac{1}{16} \cdot \frac{4}{3} \int_{-1}^{1} (1-\eta^2) \ d\eta \\
&= \frac{1}{16} \cdot \frac{4}{3} \cdot \frac{4}{3}\\
&= \frac{1}{9}
\end{align*}
Case: 3rd Off-diagonal entry ($i=0, j=3$):

Same as 1st off-diagonal by symmetry, so the result is $\frac{2}{9}$.


Substituting all these results back into the integral matrix, we have:
\[\int_{-1}^{1} \int_{-1}^{1} \Phi \cdot \Phi^T \ d\xi d\eta = 
\frac{1}{9}
\begin{bmatrix}
4 & 2 & 1 & 2 \\
2 & 4 & 2 & 1 \\
1 & 2 & 4 & 2 \\
2 & 1 & 2 & 4
\end{bmatrix}\]
Therefore, the elemental mass matrix is:
\[M^{(e)} = \frac{|x_1-x_0| \cdot |y_2-y_1|}{4} \cdot \frac{1}{9}
\begin{bmatrix}
4 & 2 & 1 & 2 \\
2 & 4 & 2 & 1 \\
1 & 2 & 4 & 2 \\
2 & 1 & 2 & 4
\end{bmatrix}\]
For our mesh, all elements are squares of side length $1$, so $|x_1 - x_0| = |y_1 - y_0| = 1$. Therefore, we have:
\[M^{(e)} = \frac{1}{36}
\begin{bmatrix}
4 & 2 & 1 & 2 \\
2 & 4 & 2 & 1 \\
1 & 2 & 4 & 2 \\
2 & 1 & 2 & 4
\end{bmatrix}\] 

Next, we compute the elemental stiffness matrix $K^{(e)}$. Denote $\nabla \Phi = [\nabla \Phi_0, \nabla \Phi_1, \nabla \Phi_2, \nabla \Phi_3]$. Note that $K^{(e)}$ can also be expressed as:
\[K^{(e)} = \int_{\Omega_e} \nabla \Phi(x,y) \cdot \nabla \Phi^T(x,y) \ dx\]
Using the change of variables, we have:
\[K^{(e)} = \int_{-1}^{1} \int_{-1}^{1} (J^{-1}\nabla \Phi(\xi,\eta)) \cdot (J^{-1} \nabla \Phi(\xi,\eta))^T |\det{J(\xi,\eta)}| \ d\xi d\eta\]

With the Jacobian determinant factored out, we have:
\[K^{(e)} = \frac{ |x_1 - x_0| \cdot |y_2 - y_1|}{4} \int_{-1}^{1} \int_{-1}^{1} (J^{-1}\nabla \Phi(\xi,\eta)) \cdot (J^{-1} \nabla \Phi(\xi,\eta))^T \ d\xi d\eta\]
As reference from question 4, we have the gradients:
\[\nabla  \Phi_0(x,y) = \frac{1}{2}\begin{bmatrix}
-(1-\eta) \\ -(1-\xi)
\end{bmatrix}, \quad \nabla  \Phi_1(x,y) = \frac{1}{2}\begin{bmatrix}
(1-\eta) \\ -(1+\xi)
\end{bmatrix}\] 
\[\nabla  \Phi_2(x,y) = \frac{1}{2}\begin{bmatrix}
(1+\eta) \\ (1+\xi)
\end{bmatrix}, \quad \nabla  \Phi_3(x,y) = \frac{1}{2}\begin{bmatrix}
-(1+\eta) \\ (1-\xi)
\end{bmatrix}\]
Note any product of the form $\nabla \Phi_i \cdot \nabla \Phi_j$ will have a factor of $\frac{1}{4}$.

Note that :
\[\nabla \Phi \cdot \nabla \Phi^T = \begin{bmatrix}
\nabla \Phi_0 \cdot \nabla \Phi_0  & \nabla \Phi_0  \cdot \nabla \Phi_1  & \nabla \Phi_0  \cdot \nabla \Phi_2  & \nabla \Phi_0  \cdot \nabla \Phi_3  \\
\nabla \Phi_1  \cdot \nabla \Phi_0  & \nabla \Phi_1  \cdot \nabla \Phi_1  & \nabla \Phi_1  \cdot \nabla \Phi_2  & \nabla \Phi_1  \cdot \nabla \Phi_3  \\
\nabla \Phi_2  \cdot \nabla \Phi_0  & \nabla \Phi_2  \cdot \nabla \Phi_1  & \nabla \Phi_2  \cdot \nabla \Phi_2  & \nabla \Phi_2  \cdot \nabla \Phi_3  \\
\nabla \Phi_3  \cdot \nabla \Phi_0  & \nabla \Phi_3  \cdot \nabla \Phi_1  & \nabla \Phi_3  \cdot \nabla \Phi_2  & \nabla \Phi_3  \cdot \nabla \Phi_3
\end{bmatrix}\]
The stiffness matrix is also symmetric, so we will only show one sample calculation for each unique entry.

Case: Main-diagonal entry ($i=j=0$):
\begin{align*}
\int_{-1}^{1} \int_{-1}^{1} \nabla \Phi_0 \cdot \nabla \Phi_0 \ d\xi d\eta &= \int_{-1}^{1} \int_{-1}^{1} \left(-\frac{1}{2}(1-\eta)\right)^2 + \left(-\frac{1}{2}(1-\xi)\right)^2 \ d\xi d\eta \\
&= \frac{1}{4} \int_{-1}^1 \int_{-1}^1 (1-\eta)^2 + (1-\xi)^2 \ d\xi d\eta \\
&=\frac{1}{4} \left[\int_{-1}^1 (1-\eta)^2 \ \int_{-1}^1 \ d\xi \ d\eta + \int_{-1}^1 (1-\xi)^2 \ \int_{-1}^1 \ d\eta \ d\xi \right] \\
&= \frac{1}{4} \left[ 2\int_{-1}^1 (1-\eta)^2 d\eta + 2\int_{-1}^1  (1-\xi)^2 d\xi \right] \\
&= \frac{1}{2} \left[ \frac{8}{3} + \frac{8}{3} \right] \\
&= \frac{1}{2} \cdot \frac{16}{3} \\
&= \frac{8}{3} \\
&= \frac{4}{6} \cdot 4
\end{align*}

Case: First Off-Diagonal entry ($i=0, j=1$):
\begin{align*}
\int_{-1}^{1} \int_{-1}^{1} \nabla \Phi_0 \cdot \nabla \Phi_1 \ d\xi d\eta &= \int_{-1}^{1} \int_{-1}^{1} \left(-\frac{1}{2}(1-\eta)\right)\left(\frac{1}{2}(1-\eta)\right) + \left(-\frac{1}{2}(1-\xi)\right)\left(-\frac{1}{2}(1+\xi)\right) \ d\xi d\eta \\
&= \frac{1}{4} \int_{-1}^1 \int_{-1}^1 -(1-\eta)^2 + (1-\xi^2) \ d\xi d\eta \\
&=\frac{1}{4} \left[\int_{-1}^1 -(1-\eta)^2 \ \int_{-1}^1 \ d\xi \ d\eta + \int_{-1}^1 (1-\xi^2) \ \int_{-1}^1 \ d\eta \ d\xi \right] \\
&= \frac{1}{4} \left[ -2\int_{-1}^1 (1-\eta)^2 d\eta + 2\int_{-1}^1  (1-\xi^2) d\xi \right] \\
&= \frac{1}{2} \left[ -\frac{8}{3} + \frac{4}{3} \right] \\
&= \frac{1}{2} \cdot \left(-\frac{4}{3}\right) \\
&= -\frac{2}{3} \\
&= \frac{4}{6} \cdot (-1)
\end{align*}

Case: Second Off-Diagonal entry ($i=0, j=2$):
\begin{align*}
    \int_{-1}^{1} \int_{-1}^{1} \nabla \Phi_0 \cdot \nabla \Phi_2 \ d\xi d\eta &= \int_{-1}^{1} \int_{-1}^{1} \left(-\frac{1}{2}(1-\eta)\right)\left(\frac{1}{2}(1+\eta)\right) + \left(-\frac{1}{2}(1-\xi)\right)\left(\frac{1}{2}(1+\xi)\right) \ d\xi d\eta \\
&= -\frac{1}{4} \int_{-1}^1 \int_{-1}^1 (1-\eta^2) + (1-\xi^2) \ d\xi d\eta \\
&=\ -\frac{1}{4} \left[\int_{-1}^1 (1-\eta^2) \ \int_{-1}^1 \ d\xi \ d\eta + \int_{-1}^1 (1-\xi^2) \ \int_{-1}^1 \ d\eta \ d\xi \right] \\
&= -\frac{1}{4} \left[ 2\int_{-1}^1 (1-\eta^2) d\eta + 2\int_{-1}^1  (1-\xi^2) d\xi \right] \\
&= -\frac{1}{2} \left[ \frac{4}{3} + \frac{4}{3} \right] \\
&= -\frac{1}{2} \cdot \frac{8}{3} \\
&= -\frac{4}{3} \\
&= \frac{4}{6} \cdot (-2)
\end{align*}

Case: Third Off-Diagonal entry ($i=0, j=3$):

Same as First Off-Diagonal, so the result is $-\frac{4}{6}$.

Substituting all these results back into the integral matrix, we have:
\[\int_{-1}^{1} \int_{-1}^{1} \nabla \Phi \cdot \nabla \Phi^T \ d\xi d\eta = 
\frac{4}{6}
\begin{bmatrix}
4 & -1 & -2 & -1 \\
-1 & 4 & -1 & -2 \\
-2 & -1 & 4 & -1 \\
-1 & -2 & -1 & 4
\end{bmatrix}\]

Therefore, the elemental stiffness matrix is:
\[K^{(e)} = \frac{|x_1 - x_0| \cdot |y_2 - y_1|}{4} \cdot \frac{4}{6}
\begin{bmatrix}
4 & -1 & -2 & -1 \\
-1 & 4 & -1 & -2 \\
-2 & -1 & 4 & -1 \\
-1 & -2 & -1 & 4
\end{bmatrix}\]

For our mesh, where all the elements are of side length $1$, we have:
\[K^{(e)} = \frac{1}{6}
\begin{bmatrix}
4 & -1 & -2 & -1 \\
-1 & 4 & -1 & -2 \\
-2 & -1 & 4 & -1 \\
-1 & -2 & -1 & 4
\end{bmatrix}\]
Note that our elemental matrices would differ based on the mesh construction due to definition of the Jacobian.

\question
Assemble the global mass matrix $M$ and global stiffness matrix $K$ for the entire $2 \times 2$ mesh using the connectivity determined in question 2. Impose homogeneous Dirichlet boundary conditions on all boundary nodes. Write the resulting discretized ODE system in the following format:
\[M \frac{dU}{dt} + KU = F(t)\]

\solution
Given the Dirichlet boundary conditions on all boundary nodes, we only need to consider the interior nodes for our global matrices, therefore our mesh is reduced to $2 \times 2$ rectangles, for a total of $4$ rectangular elements. From our initial global indexing, we reduce the problem to have $9$ interior nodes, numbered from $0$ to $8$. We'll also number our elements from $0$ to $3$ in a row-wise manner:

\begin{center}
\includegraphics[width=.75\textwidth]{updated_mesh.jpg}
\end{center}

Using the python method defined in question 2, the updated connectivities for each element are as follows:

\begin{center}
\begin{tabular}{c|c c c c}
Element & Node 0 & Node 1 & Node 2 & Node 3 \\
\hline
0 & 3 & 4 & 1 & 0 \\
1 & 4 & 5 & 2 & 1 \\
2 & 6 & 7 & 4 & 3 \\
3 & 7 & 8 & 5 & 4\\
\end{tabular}
\end{center}

We will adjust the indexing in the preliminary setup to match this scenario. From before we had:
\[\sum_{i,j=1}^{N} \left( \int_{\Omega} \Phi_i \Phi_j \ dx \right) \frac{dU_i}{dt} + \nu \sum_{i,j=1}^{N}  \left( \int_{\Omega} \nabla \Phi_i \cdot \nabla \Phi_j \ dx \right) U_i = \sum_{j=1}^{N}\int_{\Omega} f \Phi_j \ dx\] 
Was our global discretized system, where $N$ is the total number of nodes. Rewriting this terms of the elemental components, we have:
\[\left(\sum_{e=1}^{E} M^{(e)} \right) \frac{dU}{dt} + \nu \left(\sum_{e=1}^{E} K^{(e)}\right) U = \sum_{e=1}^{E} F^{(e)}(t)\]
Where $E$ is the total number of elements.

Correcting for our indexing scheme, we have:
\[\left(\sum_{e=0}^{3} M^{(e)} \right) \frac{dU}{dt} + \nu \left(\sum_{e=0}^{3} K^{(e)}\right) U = \sum_{e=0}^{3} F^{(e)}(t)\]

Such that that $M = \sum_{e=0}^{3} M^{(e)}$ and $K = \sum_{e=0}^{3} K^{(e)}$. We could lump the $\nu$ into $K$ to get the form:
\[M \frac{dU}{dt} + K U = F(t)\]
Also, since we have the homogeneous heat equation, $F^{(e)}(t) = 0$ for all $e$.

In addition to methods defined in Question 2, the following code was used to assemble our global matrices:
\begin{lstlisting}[language=Python, caption={Question 6 Python}]
import numpy as np
import sympy as sp

def det_jacobian(xe, ye):
    return (1/4) * abs(xe[1] - xe[0]) * abs(ye[2] - ye[1])

def element_mass_matrix(xe, ye):
    detJ = det_jacobian(xe, ye)
    Me = (detJ / 9) * np.array([[4, 2, 1, 2],
                                 [2, 4, 2, 1],
                                 [1, 2, 4, 2],
                                 [2, 1, 2, 4]])
    return Me

def element_stiffness_matrix(xe, ye):
    detJ = det_jacobian(xe, ye)
    Ke = detJ *(4 / 6) * np.array([[4, -1, -2, -1],
                                 [-1, 4, -1, -2],
                                 [-2, -1, 4, -1],
                                 [-1, -2, -1, 4]])
    return Ke

def generate_global_coordinates(width, height=None):
    if height is None:
        height = width
    
    # Generate global coordinates for interior nodes
    # given Dirichlet BCs
    x_nodes = np.linspace(-2, 2, width)
    x_nodes = x_nodes[1:-1]
    y_nodes = np.linspace(-2, 2, height)
    y_nodes = y_nodes[1:-1]

    # Create meshgrid of coordinates
    x_mesh, y_mesh = np.array(np.meshgrid(x_nodes, y_nodes))
    # Reshape as list of (x, y) pairs
    global_coordinates = np.column_stack((x_mesh.ravel(), y_mesh.ravel()))
    return global_coordinates


def global_assembly(width, height=None, nu=0.05):
    if height is None:
        height = width
        
    global_coordinates = generate_global_coordinates(width, height)
    
    global_indices = global_indexing(width, height)
    connectivity_matrix = generate_connectivity_matrix(global_indices)

    num_nodes = (width - 2) * (height - 2)
    M_global = np.zeros((num_nodes, num_nodes))
    K_global = np.zeros((num_nodes, num_nodes))
    
    for element in range(connectivity_matrix.shape[0]):
        # Get the global node indices for this element
        element_coordinates = connectivity_matrix[element, :]
        
        # Extract x and y coordinates for element
        # x coordinates
        xe = global_coordinates[element_coordinates, 0]  
        # y coordinates
        ye = global_coordinates[element_coordinates, 1] 
        
        # Compute element matrices
        Me = element_mass_matrix(xe, ye)
        Ke = element_stiffness_matrix(xe, ye)
        
        # Add element contributions to global matrices
        for i_local in range(4):
            i_global = element_coordinates[i_local]
            for j_local in range(4):
                j_global = element_coordinates[j_local]
                M_global[i_global, j_global] += Me[i_local, j_local]
                K_global[i_global, j_global] += Ke[i_local, j_local]
    
    return M_global, K_global

if __name__ == "__main__":
    width = 5  # Number of nodes along one dimension
    global_indices = global_indexing(width)
    connectivity_matrix = generate_connectivity_matrix(global_indices)
    M, K = global_assembly(width)

    with open("./outputs_4/matrices.txt", "w") as f:
        print(global_indices)
        latex_matrix = sp.latex(sp.Matrix(global_indices))
        f.write("Global Indices Matrix:\n")
        f.write(latex_matrix + "\n\n")
        
        print(connectivity_matrix)
        latex_matrix = sp.latex(sp.Matrix(connectivity_matrix))
        f.write("Connectivity Matrix:\n")
        f.write(latex_matrix + "\n\n")

        M = np.round(M, 4)
        latex_matrix = sp.latex(sp.Matrix(M))
        f.write("Mass Matrix:\n")
        f.write(latex_matrix + "\n\n")

        K = np.round(K, 4)
        latex_matrix = sp.latex(sp.Matrix(K))
        f.write("Stiffness Matrix:\n")
        f.write(latex_matrix + "\n\n")
\end{lstlisting}

Our mass matrix $M$ will be of size $9 \times 9$ and was found to be:
\[\left[\begin{matrix}0.1111 & 0.0556 & 0.0 & 0.0556 & 0.0278 & 0.0 & 0.0 & 0.0 & 0.0\\0.0556 & 0.2222 & 0.0556 & 0.0278 & 0.1111 & 0.0278 & 0.0 & 0.0 & 0.0\\0.0 & 0.0556 & 0.1111 & 0.0 & 0.0278 & 0.0556 & 0.0 & 0.0 & 0.0\\0.0556 & 0.0278 & 0.0 & 0.2222 & 0.1111 & 0.0 & 0.0556 & 0.0278 & 0.0\\0.0278 & 0.1111 & 0.0278 & 0.1111 & 0.4444 & 0.1111 & 0.0278 & 0.1111 & 0.0278\\0.0 & 0.0278 & 0.0556 & 0.0 & 0.1111 & 0.2222 & 0.0 & 0.0278 & 0.0556\\0.0 & 0.0 & 0.0 & 0.0556 & 0.0278 & 0.0 & 0.1111 & 0.0556 & 0.0\\0.0 & 0.0 & 0.0 & 0.0278 & 0.1111 & 0.0278 & 0.0556 & 0.2222 & 0.0556\\0.0 & 0.0 & 0.0 & 0.0 & 0.0278 & 0.0556 & 0.0 & 0.0556 & 0.1111\end{matrix}\right]\]

Our stiffness matrix $K$ will also be of size $9 \times 9$ and was found to be (without $\nu$):
\[\left[\begin{matrix}0.6667 & -0.1667 & 0.0 & -0.1667 & -0.3333 & 0.0 & 0.0 & 0.0 & 0.0\\-0.1667 & 1.3333 & -0.1667 & -0.3333 & -0.3333 & -0.3333 & 0.0 & 0.0 & 0.0\\0.0 & -0.1667 & 0.6667 & 0.0 & -0.3333 & -0.1667 & 0.0 & 0.0 & 0.0\\-0.1667 & -0.3333 & 0.0 & 1.3333 & -0.3333 & 0.0 & -0.1667 & -0.3333 & 0.0\\-0.3333 & -0.3333 & -0.3333 & -0.3333 & 2.6667 & -0.3333 & -0.3333 & -0.3333 & -0.3333\\0.0 & -0.3333 & -0.1667 & 0.0 & -0.3333 & 1.3333 & 0.0 & -0.3333 & -0.1667\\0.0 & 0.0 & 0.0 & -0.1667 & -0.3333 & 0.0 & 0.6667 & -0.1667 & 0.0\\0.0 & 0.0 & 0.0 & -0.3333 & -0.3333 & -0.3333 & -0.1667 & 1.3333 & -0.1667\\0.0 & 0.0 & 0.0 & 0.0 & -0.3333 & -0.1667 & 0.0 & -0.1667 & 0.6667\end{matrix}\right]\]

As expected, both $M$ and $K$ are symmetric, banded matrices.

\question
Using the following initial condition
\[u(x,y,0) = \sin(2 \pi x) \sin(2 \pi y)\]
to solve the reduced ODE system from $t=0$ to $t=1$. (Show Code)

\solution
Given our system:
\[M \frac{dU}{dt} + \nu KU = 0\]
We can rearrange this to get:
\[\frac{dU}{dt} = -\nu M^{-1}K U\]

{\large \textbf{Explicit: Forward Euler}}\\
Let us find $U^{n+1}$ explicitly using forward difference in time:
\[\frac{U^{n+1} - U^{n}}{\Delta t} = -\nu M^{-1}K U^{n}\]
\[U^{n+1} = U^{n} - \nu \Delta t  M^{-1}K U^{n}\]
Now let us derive the CFL condition for stability. Note that we may rewrite our scheme above as:
\[U^{n+1} = \left( I - \nu \Delta t M^{-1}K \right) U^{n}\]
Let $A = M^{-1}K$ and $B = I - \nu \Delta t A$. For stability, we require that the spectral radius of $B$ be less than or equal to $1$:
\[\rho(B) = \max{|\lambda(B)|} \leq 1\]
Where the maximum is taken over all eigenvalues $\lambda$ of $B$. Note that each eigenvalue of $B$ are related to a corresponding eigenvalue of $A$ as follows:
\[\lambda(B) = 1 - \nu \Delta t \lambda(A)\]
Therefore, we require:
\[\max{|1 - \nu \Delta t \lambda(A)|} \leq 1\]
Which implies that for all eigenvalues $\lambda(A)$:
\[|1 - \nu \Delta t \lambda(A)| \leq 1\]
We note that $A = M^{-1}K$ is positive definite (has real, positive eigenvalues). Through a quick check using Python, the list of eigenvalues of $A$ are:
\[\left[\begin{matrix}24.0\\1.17981541490574 \cdot 10^{-15}\\3.0\\6.0\\15.0\\12.0\\3.0\\12.0\\15.0\end{matrix}\right]\]
Note that $\rho(A) = 24$. 

So given that all eigenvalues of $A$ are non-negative, to satisfy our stability condition, we require:
\[|1 - \nu \Delta t \lambda(A)| \leq 1\]
Which implies:
\[ -1 \leq 1 - \nu \Delta t \lambda(A) \leq 1\]
We may bound these inequalities separately. The right inequality simplifies to:
\[0 \leq \nu \Delta t \lambda(A)\]
Rearranging for $\Delta t$, we have:
\[\Delta t \geq 0\]
Which is vacuously true as heat is unstable in backwards time. The left inequality simplifies to:
\[-2 \leq - \nu \Delta t \lambda(A)\]
Rearranging for $\Delta t$, we have:
\[\Delta t \leq \frac{2}{\nu \lambda(A)}\]
To satisfy this for all eigenvalues of $A$, we use the maximum eigenvalue of A $\lambda_{max}(A) = 24$:
\[\Delta t \leq \frac{2}{\nu \lambda_{max}(A)} = \frac{2}{0.05 \cdot 24} = \frac{5}{3} \approx 1.6667\]
Therefore we may use the explicit scheme so long as $\Delta t \leq \frac{5}{3}$ for stability.

The following code was used to implement the explicit forward Euler method:
\lstinputlisting[language=Python, caption={explicit\_solver.py}]{../explicit_solver.py}

{\large \textbf{Semi-Implicit: Crank-Nicolson Method}}\\
Let us use Crank-Nicolson method to solve this system.

The forward Euler step is:
\[M\frac{U^{n+1}-U^{n}}{\Delta t} = F^{n}(U) = -K U^{n}\]
The backward Euler step is:
\[ M \frac{U^{n+1}-U^{n}}{\Delta t} = F^{n+1}(U) = -K U^{n+1}\]
Using Crank-Nicolson, we average these two steps:
\[M\frac{U^{n+1}-U^{n}}{\Delta t} = \frac{1}{2} \left( F^{n}(U) + F^{n+1}(U) \right) = -\frac{1}{2} K \left( U^n + U^{n+1} \right)\]
Rearranging, we have:
\[M( U^{n+1}-U^{n}) = -\frac{\Delta t}{2} K \left( U^n + U^{n+1} \right)\]
\[\left( M + \frac{\Delta t}{2} K \right) U^{n+1} = \left( M - \frac{\Delta t}{2} K \right) U^{n}\]

\question
Write your own solver for solving linear systems. You can freely choose the methods from Gaussian, Jacobi, or Gauss-Seidel.
\solution
\emph{Extra: See Appendix for past Maple implementations of Gaussian Elimination, Jacobi, and Gauss-Seidel.}

\question
Perform a convergence study with different refinements on time steps. Plot the log-log plot of error vs time step size. 

\question
Plot $U(t)$ at $T=1$.

\newpage
    \vspace{1em}
    {\Large\textbf{Appendix}}

\section{Maple Implementations of Linear Solvers}

These are implementations of Gaussian Elimination, Jacobi Method, and Gauss-Seidel Method in Maple I have done in the past.

These were the questions being answered:
\begin{center}
\includegraphics[width=.9\textwidth]{questions.jpg}
\end{center}
The following pages contain the Maple implementations and sample outputs. 

Note: the Matrix Solver is Gaussian Elimination with addition of LU decomposition. No row exchanges were implemented.
\includepdf[pages=-]{maple.pdf}

\section{Assignment Code}
\startappendix

\end{document}